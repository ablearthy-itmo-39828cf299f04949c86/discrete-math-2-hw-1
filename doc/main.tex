\documentclass[12pt, a4paper]{article}

\usepackage{mypreamble}
\usepackage{standalone}

\usepackage{pdflscape}

\title{Discrete Math II, HW I}
\date{March, 2023}
\author{Aminev Timur, M3104}


\newcommand\EGraph{\mathcal{G}^{*}}
\newcommand\EGraphL{\mathcal{G}}

\begin{document}
\maketitle

\problem The graph of Europe \(\EGraph = \Pair{V, E}\). Let \(\EGraphL\) be the
largest connected component of \(\EGraph\)

\begin{enumerate}[label=\alph*)]
\item \Cref{fig:map_basic} represents \(\EGraph\) with \textbf{no} intersecting
edges
\item \(|V| = 49\), \(|E| = 92\)

In order to count degree of each node, we should go through all edges, and
increment degree of a vertex if the edge contains the vertex. %TODO: link to code
\begin{align*}
\delta(\EGraphL) &= 1 \\
\Delta(\EGraphL) &= 9
\end{align*}

Radius, diameter and center can be obtained by calculating eccentricity for
each vertex (using BFS)

\begin{align*}
\rad(\EGraphL) &= 5 \\
\diam(\EGraphL) &= 8 \\
\mycenter(\EGraphL) &= \{\texttt{RUS}, \texttt{BLR}, \texttt{LTU}, \texttt{POL}, \texttt{UKR}, \texttt{CZE}, \texttt{DEU}, \texttt{SVK}, \texttt{HUN}, \texttt{HRV}, \texttt{AUT}, \texttt{CHE}, \texttt{SVN}\}
\end{align*}

\texttt{Girth} equals 3 --- \(\{\texttt{CHE}, \texttt{AUT}, \texttt{LIE}\}\)

Vertex and edges connectivity are obtained from the following observation:
\texttt{DNK} is conneced only with \texttt{DEU}, therefore
\[\varkappa(\EGraphL) = \lambda(\EGraphL) = 1\]

\item \textbf{Minimum vertex coloring}. Due to \textit{four color theorem},
there're at most 4 colors needed. Let's explore is it possible to use less
colors? Let's try to use 3 colors. However, it's impossible to color countries
\texttt{HRV}, \texttt{SRB}, \texttt{BIH}, \texttt{MNE}. They're all connected
with each other, so three colors is not enough, therefore we have to use four
colors. I used SMT solver (Z3) to find function \(V \to \mathbb{N}\). Here's
\href{https://github.com/ablearthy-itmo-39828cf299f04949c86/discrete-math-2-hw-1/blob/753891b/auto/vertex_coloring.py}{the
script}. See \cref{fig:vertex_coloring_map}.

\item \textbf{Minimum edge coloring}. I used SMT solver (Z3). Here's
\href{https://github.com/ablearthy-itmo-39828cf299f04949c86/discrete-math-2-hw-1/blob/8bc76f0/auto/edge_coloring.py}{the
solver}.
\textit{9} colors is enough. Coloring is shown in \cref{fig:edge_coloring_map}.

\item \textbf{Find the maximum clique}. I wrote the script to find maximum
clique
\href{https://github.com/ablearthy-itmo-39828cf299f04949c86/discrete-math-2-hw-1/blob/1c391015b3b3a8822ac5e7c6f949004ca919e267/auto/find_clique.py}{\texttt{find\_clique.py}}

The maximum clique is of size 4.
\(\texttt{HRV}, \texttt{SRB}, \texttt{BIH}, \texttt{MNE}\)

\item \textbf{Find the maximum stable set}. \href{https://github.com/ablearthy-itmo-39828cf299f04949c86/discrete-math-2-hw-1/blob/79a6584/auto/stable_set.py}{The script}.
\[|S| = 19\]
Countries included in stable set are shown in \cref{fig:stable_set_map}.

\item \textbf{Find the maximum matching}.
\href{https://github.com/ablearthy-itmo-39828cf299f04949c86/discrete-math-2-hw-1/blob/1fae45a/auto/matching.py}{The
script}.
\[|M| = 20\]
Countries included in stable set are shown in \cref{fig:matching_map}.

\item \textbf{Find the minimum vertex cover}.
\href{https://github.com/ablearthy-itmo-39828cf299f04949c86/discrete-math-2-hw-1/blob/cb98748/auto/vertex_cover.py}{The
script}.
\[|R| = 25\]
Countries included in vertex cover are shown in \cref{fig:vertex_cover_map}.

\item \textbf{Find the minimum edge cover}.
\href{https://github.com/ablearthy-itmo-39828cf299f04949c86/discrete-math-2-hw-1/blob/90b7019/auto/edge_cover.py}{The
script}.
\[|F| = 24\]
Edges included in edge cover are shown in \cref{fig:edge_cover_map}.




\end{enumerate}



\begin{landscape}
\begin{figure}
\centering
\includestandalone[mode=build,width=0.95\linewidth]{maps/basic}
\caption{Basic map}\label{fig:map_basic}
\end{figure}
\end{landscape}

\begin{landscape}
\begin{figure}
\centering
\includestandalone[mode=build,width=0.95\linewidth]{maps/vertex_coloring}
\caption{Vertex coloring}\label{fig:vertex_coloring_map}
\end{figure}
\end{landscape}

\begin{landscape}
\begin{figure}
\centering
\includestandalone[mode=build,width=0.95\linewidth]{maps/edge_coloring}
\caption{Edge coloring}\label{fig:edge_coloring_map}
\end{figure}
\end{landscape}

\begin{landscape}
\begin{figure}
\centering
\includestandalone[mode=build,width=0.95\linewidth]{maps/stable_set}
\caption{Maximum stable set}\label{fig:stable_set_map}
\end{figure}
\end{landscape}

\begin{landscape}
\begin{figure}
\centering
\includestandalone[mode=build,width=0.95\linewidth]{maps/matching}
\caption{Maximum matching}\label{fig:matching_map}
\end{figure}
\end{landscape}

\begin{landscape}
\begin{figure}
\centering
\includestandalone[mode=build,width=0.95\linewidth]{maps/vertex_cover}
\caption{Minimum vertex cover}\label{fig:vertex_cover_map}
\end{figure}
\end{landscape}

\begin{landscape}
\begin{figure}
\centering
\includestandalone[mode=build,width=0.95\linewidth]{maps/edge_cover}
\caption{Minimum edge cover}\label{fig:edge_cover_map}
\end{figure}
\end{landscape}
\end{document}
